\section{Turns}
During combat, a single \textbf{turn} is roughly equivalent to six seconds in the game world. Each turn is subdivided into two phases: the \textbf{planning phase}, and the \textbf{execution phase}. During the planning phase, each of the two opposing sides plan out their actions. During the execution phase, both side's planned actions are simultaneously executed, with differing success. Generally, during the execution phase, all action by player characters and non-player characters alike is involuntary. When two characters decide to attack each other at the same body part during the planning phase, it is very likely during the execution phase that one of them will parry the attack, effectively wasting that turn.\newline
Should a combatant commit a particularly punishable move, such as a clumsy attack \textit{(failed roll against dexterity)}, the target has to take a roll for willpower. Should the roll fail, the character will commit an involuntary action: instinctively counterattack. Should the roll be succesfull, the character's controller can decide whether they wish to counterattack or not, and if yes, then how exactly, effectively granting them an extra turn.\newline
A turn can also be spent by simply moving rather than attacking or casting a spell. However, this tabletop roleplaying system does not define any system for mapping movement, attributes and turns together, so players have to make up house rules for defining movement on a grid-based system. Under house rules, a turn can also be spent doing other potentially important actions, but hostile combatants may in turn intercept said actions and attack the character.\newline
Since combat happens on two phases and players may not be able to necessarily anticipate hostile actions during the planning phase, they may give conditional orders to their own characters, such as \textit{"retaliate if attacked (in melee)"} or \textit{"attempt to parry or block attacks but do not retaliate if attacked"}.\newline
To sum up:
\begin{enumerate}
  \item \textbf{Planning phase:} Each side decides what actions their characters intend to commit during this turn. Characters with high dexterity may plan more actions, as they may be fast enough to actually complete them.
  \item \textbf{Execution phase:} Each character attempts to commit their actions. Since targets and action-commiters will inevitably overlap, the intended actions of certain characters will be intercepted, causing a roll againt willpower to decide if the interrupted character's next action will be involuntary or player-controlled.
  \item The results of the turn are evaluated before the next turn may begin. Any damage done to characters is documented. Characters with negative status effects that prevent action are forced to skip the next turn. Damage done by negative status effects like poison and bleeding is evaluated.
  \item The turn is over, and the planning phase of the next turn may begin.
\end{enumerate}
During each attack - melee or ranged - the character's dexterity is measured up with a randomly generated or dice-rolled number.
\begin{itemize}
\item \textbf{Targeted Strike:} The character targets a specific body part of the enemy. On every strike, they roll either two 10-sided dice or four 5-sided dice \textit{(or use a random number generator to generate a number between 0 and 20)}, where randomness is measured up to the character's \textbf{Dexterity} stat. If the number is below the character's dexterity minus four, the strike is a success and hits the intended body part \textit{(if it was a targeted strike)} - if it's between dexterity minus four and dexterity, it's a \textit{"bittersweet success"} and the character hits a neighbouring body part instead \textit{(in case of torso: the neighbour is randomly picked, RNG or single 6-sided dice roll)}. If the rolled number is above the character's dexterity, it's a miss.
\item \textbf{Untargeted Strike:} The character simply swings their weapon in the enemy's general direction. A random number is generated or rolled between 0 and 20. If the character rolls below their dexterity plus two, it is a success, if over their dexterity plus two, a failure. The body part that gets damaged is decided randomly - which is discussed at the \textbf{Body parts} section of this document later on.
\end{itemize}
By default, involuntary counterattacks are always untargeted.
