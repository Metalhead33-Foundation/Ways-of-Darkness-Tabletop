\section{Damage}
During combat, it is inevitable, that characters will sustain damage, sometimes even lethal damage. However, it is important to distinguish between different types of damage characters may sustain, as they may different kinds of resistant to different types of damage. Keep in mind, that most weapons and offensive spells typically simultaneously inflict two or more types of damage!
\begin{itemize}
\item \textbf{Physical:} The most common type of damage, this encapsulates all forms of \textit{"mundane"} types of damage that can befall upon mere mortals, anything that doesn't involve magic.
	\begin{itemize}[label=$\star$]
		\item \textbf{Crushing:} Damage done to entities via applying blunt force to a relatively large area on the target. Since every weapon in existence has weight behind it, every weapon - at least, every melee weapon - inflicts some level of crushing damage upon the intended target, even if we're talking about sharp blades. Fists, quarterstaffs, cudgels, hammers and maces inflict only crushing damage. Among ranged weapons, slings cause crushing damage. Depending on the type of ammunition \textit{(chiefly the shape of the bullet)}, firearms can cause both crushing and piercing damage.
		\item \textbf{Slashing:} Damage done to entities via severing tissues, creating cuts. This type of damage is perhaps the most deadly to unarmoured and naked people, but is not much of a worry when the target is wearing armour, as most types of armour grant near-complete immunity to slashing damage. Among other things, bladed weapons \textit{(swords, daggers, scimitars, sabres, etc.)} and axes deal slashing damage when swung against the target.
		\item \textbf{Piercing:} Damage done to applying pressure onto a rather small area, a small point on the target. This type of attack is very affective against not just unarmoured opponents, but also opponents wearing chainmail or any kind of armour not made out of solid metal. Plate mail, breastplates and other more solid metal armour however offers near-complete immunity to piercing attacks. Among melee weapons, piercing damage is primarily caused by spears and pikes, but can also be caused by bladed weapons of any kind, when they are used in certain ways \textit{(stabbing with knives and daggers, half-swording with swords, etc.)}. Among ranged weapons, javelins, bows and crossbows cause piercing damage.  Depending on the type of ammunition \textit{(chiefly the shape of the bullet)}, firearms can cause both crushing and piercing damage.
		\item \textbf{Thermal:} Damage done to entities via applying heat that's beyond what their body is capable of tolerating, whether too cold or too hot. Cold thermal attacks cause frostbite, while hot thermal attacks cause burns. Thermal damage can be dealt not just by hostile entities, but by the environment itself, or even a character's own equipment \textit{(chiefly their clothing or armour)}.
			\begin{itemize}[label=$\star$]
				\item \textbf{Cold}
				\item \textbf{Hot}
			\end{itemize}
		\item \textbf{Biological:} Damage done to entities via various biological processes, such as poison or disease.
	\end{itemize}
\item \textbf{Magical:}
	\begin{itemize}[label=$\star$]
		\item \textbf{Energy:} Damage done to entities via exposing them to destructive spells made out of pure magical energy. Even though this description sounds rather intimidating, such spells tend to be the weakest of the destruction spells. The only way to resist them is to have Magic Resistance.
		\item \textbf{Elemental:} Subdivided into four more groups \textit{(Earth, Air, Water, Fire)}, elemental magical damage represents damage done via Elemental Magic. Resistance depends on both general Magic Resistance, as well as Resistance to that specific element.
			\begin{itemize}[label=$\star$]
				\item \textbf{Earth:} Usually co-occours with Crushing damage.
				\item \textbf{Air:} Usually co-occours with Crushing, Slashing or Piercing damage.
				\item \textbf{Water:} Usually co-occours with Cold Thermal damage and Crushing damage.
				\item \textbf{Fire:} Usually co-occours with Hot Thermal damage, sometimes with Crushing damage.
			\end{itemize}
	\end{itemize}
\end{itemize}
