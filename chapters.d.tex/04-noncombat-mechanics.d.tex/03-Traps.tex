\section{Traps}
It is inevitable that as adventurers explore castle ruins and age-old dungeons, they will run into traps the villains have set for them. These traps will vary from mildly annoying to outright deadly. So how will adventurers avoid them?

The answer is simple: by a game mechanic. By default, characters cannot detect any traps other than the most obvious ones \Parentheses{interpretation up to the GM/DM}, and cannot disarm any traps, meaning that once they detected a trap in the first place, their only course of action is to try to walk around them and not step into them, if possible. But with a certain feat - called \SoCalled{Security} in the main archmodule - not only their ability to detect traps improve, but they also gain the ability to disarm traps.

When disarming traps, a d20 dice \Parentheses{or the sum of four d5 dices, or five d4 dices} must be rolled - or alternatively, for videogames, a random number between 1and 20 must be generated. When the received number is smaller than the character's dexterity, the disarming is a success, and the character can put the disarmed trap into their inventory if they want, potentially rearming it at will. If the disarming attempt is a failure, the trap activates and does its damage to the one who attempted to disarm it.

Naturally, at DM discretion, constant numbers can be added to or detracted from the aforementioned randomly generated number - or it can be even multiplied or divided, to simulate trap that are harder or easier to disarm.
