\section{Enchantment}
Enchantments put various magical effects on items - usually weapons and armour, but literally any kind of items can be enchanted. The effects can vary from passive stat boosts to activated effects - even granting the one who has equipped the item the ability to cast a spell from that item is not out of the realms of possibility, even if the equiper can't do any magic otherwise.

Naturally, characters by default cannot put any enchantments on their items - they need a feat that enables that ability. Once enabled, characters are met with the ability to enchant items, with enchantments being divided into three classes: \textbf{Class C} \Parentheses{the weakest}, \textbf{Class B} \Parentheses{stronger} and \textbf{Class A} \Parentheses{the strongest} enchantments.

While modules will provide some example enchantments, it will be largely up to DM discretion to decide what kind of enchantment belongs to which class. Additionally, some items may be cursed! That is, some enchanted items might provide negative effects on its wearer/user instead of positive ones.
