\section{Lockpicking}
From time to time, adventurers will run into locked doors, and unless there is a magician in the party who can cast a spell that opens the lock, the lock will have to be picked, unless they found the key earlier. This is where lockpicking comes in.

Normally, characters cannot pick locks - that is, not without a feat that enables lockpicking. In the main archmodule, this is enabled by the Lockpicking feat. When lockpicking is enabled for a character, they can attempt to pick a lock, during which they must roll either a single 20-sided dice, four 5-sided dices or five 4-sided dices - or, in a videogame, a random number must be generated between 1 and 20. This number is measured against the character's dexterity - if the character's dexterity is greater than or equal than this random number, the lockpicking is a success. Otherwise, it is a failure. The aforementioned random number is also manipulated the tier of the lockpicker's feat, as well as other circumstances: DMs can also add constant numbers to \Parentheses{or detract constant numbers from} the random number, or even multiply or divide them to simulate more difficult-to-pick locks.

Lockpicking can only be done with \SoCalled{picks} - failing to pick a lock means losing that pick.

\small{\textbf{Note to DMs/GMs and game developers alike:} Every door with a mundane, non-magical lock should be pickable by a sufficiently good lockpicker. Making a door only openable by a key and its lock unpickable as a means of \href{https://tvtropes.org/pmwiki/pmwiki.php/Main/Railroading}{railroading} is highly discouraged by me. I am already against railroading to begin with, but if you must railroad, invent a more convenient and believable excuse than somehow inexplicably have a mundane lock unpickable, and only openable by a key even if the party has the best lockpicker in the world.}
