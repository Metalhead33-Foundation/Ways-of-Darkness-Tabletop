\subsection{Bladed Weapons}

\begin{longtable}{|C{2cm} L{2cm} L{2cm} L{8cm}|}
\hline
\large{\textbf{Name}} &
\large{\textbf{Cost}} &
\large{\textbf{Handedness}} &
\large{\textbf{Damage}}
\\ \hline
\WeaponRow{Dagger}{65 m}{One}{
\textbf{When slashing:} \hfill \break
\textit{Slashing:} $\frac{2d6 \pm modifiers}{5}*Dexterity$ \hfill \break
\textit{Crushing:} $\frac{1d4 \pm modifiers}{5}*Dexterity$ \hfill \break
\textbf{When stabbing:} \hfill \break
\textit{Piercing:} $\frac{2d8 \pm modifiers}{5}*Dexterity$ \hfill \break
\textit{Crushing:} $\frac{1d4 \pm modifiers}{5}*Dexterity$
}{Daggers have short, typically straight blades - easily concealed, they are perfect weapons for assassins and thieves, and also have plenty of utility outside of combat as well.}
\WeaponRow{Straight Short Sword}{90 m}{One}{
\textbf{When slashing:} \hfill \break
\textit{Slashing:} $\frac{2d8 \pm modifiers}{5}*Dexterity$ \hfill \break
\textit{Crushing:} $\frac{1d6 \pm modifiers}{5}*Dexterity$ \hfill \break
\textbf{When stabbing:} \hfill \break
\textit{Piercing:} $\frac{2d8 \pm modifiers}{5}*Dexterity$ \hfill \break
\textit{Crushing:} $\frac{1d6 \pm modifiers}{5}*Dexterity$
}{Straight-bladed short swords have a blade length between 30 and 60 cm \Parentheses{1 and 2 ft}. They pack a bigger punch than daggers, but also have reduced utility outside of combat. A straight-bladed short sword may also be called a \textbf{Long Seax}, a \textbf{Gladius}, a \textbf{Xiphos} or just a \textbf{Shortsword}.}
\WeaponRow{Curved Short Sword}{95 m}{One}{
\textbf{When slashing:} \hfill \break
\textit{Slashing:} $\frac{2d10 \pm modifiers}{5}*Dexterity$ \hfill \break
\textit{Crushing:} $\frac{1d8 \pm modifiers}{5}*Dexterity$ \hfill \break
\textbf{When stabbing:} \hfill \break
\textit{Piercing:} $\frac{2d4 \pm modifiers}{5}*Dexterity$ \hfill \break
\textit{Crushing:} $\frac{1d4 \pm modifiers}{5}*Dexterity$
}{Curved-bladed short swords have a blade length between 30 and 60 cm \Parentheses{1 and 2 ft}. Weapons dedicated purely for combat, their curved blades are perfect for slashing, but are rather poor for stabbing. A curved-bladed short sword may also be known as a \textbf{Short Cutlass}, a \textbf{Short Scimitar}, a \textbf{Shamshir}, a \textbf{Nimcha}, a \textbf{Dao} or a \textbf{Wakizashi}.}
\WeaponRow{Straight Medium Sword}{160 m}{One / Two}{
\textbf{When slashing:} \hfill \break
\textit{Slashing:} $\frac{2d10 \pm modifiers}{5}*Dexterity$ \hfill \break
\textit{Crushing:} $\frac{1d8 \pm modifiers}{5}*Dexterity$ \hfill \break
\textbf{When stabbing:} \hfill \break
\textit{Piercing:} $\frac{2d10 \pm modifiers}{5}*Dexterity$ \hfill \break
\textit{Crushing:} $\frac{1d8 \pm modifiers}{5}*Dexterity$ 
}{Straight-bladed medium swords have a blade length between 70 and 80 cm \Parentheses{2.2 and 2.6 ft}. A weapon dedicated purely for combat, a straight-bladed medium sword was typically called a \textbf{Jian}, an \textbf{Arming Sword} or just a \textbf{Sword}.}
\WeaponRow{Curved Medium Sword}{170 m}{One / Two}{
\textbf{When slashing:} \hfill \break
\textit{Slashing:} $\frac{2d12 \pm modifiers}{5}*Dexterity$ \hfill \break
\textit{Crushing:} $\frac{1d10 \pm modifiers}{5}*Dexterity$ \hfill \break
\textbf{When stabbing:} \hfill \break
\textit{Piercing:} $\frac{2d6 \pm modifiers}{5}*Dexterity$ \hfill \break
\textit{Crushing:} $\frac{1d4 \pm modifiers}{5}*Dexterity$ 
}{Curved-bladed medium swords have a blade length between 70 and 80 cm \Parentheses{2.2 and 2.6 ft}. A weapon dedicated purely for combat, a curved-bladed medium sword was typically called a \textbf{Scimitar}, a \textbf{Sabre}, a \textbf{Falchion}, a \textbf{Cutlass}, a \textbf{Yatagan} or a \textbf{Katana}.}
\WeaponRow{Straight Long Sword}{210 m}{Two}{
\textbf{When slashing:} \hfill \break
\textit{Slashing:} $\frac{2d12 \pm modifiers}{5}*Dexterity$ \hfill \break
\textit{Crushing:} $\frac{1d10 \pm modifiers}{5}*Dexterity$ \hfill \break
\textbf{When stabbing:} \hfill \break
\textit{Piercing:} $\frac{2d12 \pm modifiers}{5}*Dexterity$ \hfill \break
\textit{Crushing:} $\frac{1d10 \pm modifiers}{5}*Dexterity$ 
}{Straight-bladed long swords have a blade length longer than 90 \Parentheses{2.9 ft}, but seldom longer than 120 cm \Parentheses{3.9 ft}. A weapon dedicated purely for combat, a straight-bladed long sword was typically called an \textbf{Longsword}, a \textbf{Claymore} or a \textbf{Zweihänder}.}
\WeaponRow{Curved Long Sword}{220 m}{Two}{
\textbf{When slashing:} \hfill \break
\textit{Slashing:} $\frac{3d10 \pm modifiers}{5}*Dexterity$ \hfill \break
\textit{Crushing:} $\frac{1d12 \pm modifiers}{5}*Dexterity$ \hfill \break
\textbf{When stabbing:} \hfill \break
\textit{Piercing:} $\frac{2d8 \pm modifiers}{5}*Dexterity$ \hfill \break
\textit{Crushing:} $\frac{1d6 \pm modifiers}{5}*Dexterity$ 
}{Curved-bladed long swords have a blade length longer than 90 \Parentheses{2.9 ft}, but seldom longer than 120 cm \Parentheses{3.9 ft}. A weapon dedicated purely for combat, a curved-bladed long sword was typically called an \textbf{Changdao}, an \textbf{Ōdachi} or a \textbf{Nodachi}.}
\end{longtable}
