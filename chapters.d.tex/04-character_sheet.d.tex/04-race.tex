\section{Race}
A character's \textbf{race} determines quite a lot of things about a character \textemdash their expected appearence, the number of sexes they can \textit{"choose from"}, the maximums and minimums for the various attributes, the number of body parts they have \textit{(each body part has its own counter of health points)}, etc. Some backgrounds are restricted to certain races.\newline
Characters may also have a secondary race, which is typically something they assumed during their life, while their primarily race is what they were born as. For example, for a vampire who was born a human, their primary race would be Human, with their secondary race being Vampire.\newline
Races also have a variable called an \textbf{ageing factor}, which determines how long is the typical lifespan of a race compared to humans. When calculating a character's biological age in days \textemdash so long as the character is older than 6574.5 days \textemdash we must take their chronological age, reduce it by 6574.5, divide by the race's ageing factor, then increment by 6574.5. In other words: \[\frac{chronologicalAge-6574.5}{ageingFactor}+6574.5\]
It is recommended for programmers to implement secondary races dynamically generating them from primary races and having a pointer variable to the parent race. 
