\section{Religions}
Religion has always played a central role in the life of every civilization's history, and continues to do so, even if we, as individuals living in the modern age mistakenly think otherwise. Religions shape the identities, morals and various other cultural aspects of even otherwise fully secular societies, and historically have even been the cause of political events and military conflicts.

Since World of Artograch is primarily intended to be played in a fantasy setting inspired by the Middle Ages and Antiquity, religion plays a very overt and visible role in people's lives - not to mention, that in a fantasy setting where priests are also users of Clerical Magic, religion gets to be tied directly to magic - or at least, Clerical Magic.

\subsection{Religious attributes}
Religions have the following attributes:
\begin{itemize}
\item \textbf{The religions's name}
\item \textbf{The religions's virtues:} These determine what kind of actions are approved by the deity or deities of the religion, thus letting the character gain back favour, in case they previously incurred the wrath of the deitiy or deities. A religion's virtues also directly determine which Schools of Magic are open to Clerical Magic spellcasters following the religion. For some polytheistic religions, deities may have so divergent sets of values, that each have their own set of virtues, rather than one array for the whole religion.
\item \textbf{The religions's hierarchy of sins:} These determine what kind of actions are necessary to be taken to incur the wrath of the religion's deity or deities. Commiting major sins results in loss of divine favour, which has serious consequences for spellcasters of the Clerical variety - while gameplay-wise, these have no affects on non-spellcasters, when roleplaying as an otherwise religious character, they should feel guilt when commiting sins. For some polytheistic religions, deities may have so divergent sets of values, that each have their own Hierarchy of Sins, rather than one for the whole religion.
\end{itemize}
\subsection{Religion on the Character Sheet}
\begin{itemize}
\item \textbf{Religion Name:} Which is \textit{"N/A"} if the character is an Atheist. Otherwise, it's the name of a defined religion.
\item \textbf{Divine Favour:} Which is a \textit{"N/A"} if the character is an Atheist - otherwise an integer between 0 and 10, where higher numbers correspond to more saintly behaviour, while lower numbers correspond to more sinful behaviour. Breaking of the religion's rules causes the character's Divine Favour to plummet - the lower the value, the greater the sin has to be to lower it even further \Parentheses{see the religion's Hierarchy of Sins}. Reaching a Divine Favour of 0 should permanently strip a user of Clerical Magic from their powers. If a character follows a polytheistic religion where different deities have drastically different sets of values, their divien favour is tracked separately for each deity - albeit the only relevant one is the one the character is a preist of, if he/she is a user of Clerical Magic.
\end{itemize}
%\begin{table}[]
\begin{tabular}{|c|c|}
\hline
\multicolumn{2}{|c|}{\textbf{Divine Favour}} \\ \hline
10 & Messianic \\ \hline
9 & Saintly \\ \hline
8 & Examplary \\ \hline
7 & Virtuous \\ \hline
6 & Devout \\ \hline
5 & Lukewarm \\ \hline
4 & Lapsed \\ \hline
3 & Sinful \\ \hline
2 & Decadent \\ \hline
1 & Degenerate \\ \hline
0 & Damned \\ \hline
\end{tabular}
%\end{table}
