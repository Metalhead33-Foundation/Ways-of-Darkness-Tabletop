\section{Alignments}
\begin{figure}[h]
\captionsetup{justification=centering}
\begin{center}
  \begin{tikzpicture}
  \begin{scope}[x=4cm, y=4cm]
    \tkzInit[xmax=1,ymax=1,xmin=-1,ymin=-1]
  \tkzGrid
          \tkzDrawX[step=0.1]
        \tkzDrawY[step=0.1]
        \tkzLabelX[orig=false,label options={font=\tiny},step=0.33]
        \tkzLabelY[orig=false,label options={font=\tiny},step=0.33]
        \draw (0,1)  node[above]                 {Lawful}  -- (0,-1) node[below]                 {Chaotic}
          (-1,0) node[xshift=-6pt,rotate=90] {Evil} -- (1,0)  node[xshift=6pt,rotate=-90] {Good};
    \node at (-0.75,0.75) {Lawful Evil};
    \node[below] at (-0.75,0) {Neutral Evil};
    \node at (-0.75,-0.75) {Chaotic Evil};
    \node at (0.75,0.75) {Lawful Good};
    \node[below] at (0.75,0) {Neutral Good};
    \node at (0.75,-0.75) {Chaotic Good};
    \node at (0,0.75) {Lawful Neutral};
    \node[below] at (0,0) {True Neutral};
    \node at (0,-0.75) {Chaotic Neutral};
    \end{scope}
  \end{tikzpicture}
  \caption{The alignment coordinate system, or alignment compass.}
  \end{center}
\end{figure}
Rather than a strict categorization of Good vs Neutral vs Evil and Lawful vs Neutral vs Chaotic, these two axes \textit{(Good vs Evil, Lawful vs Chaotic)} are tracked on a floating-point coordinate system, where both the X axis \textit{(Good vs Evil)} and Y axis \textit{(Lawful vs Chaotic)} go from 1.0 to -1.0, with 0.0 at both axes representing true neutrality. This is done, so that alignment shifts can be tracked. Nevertheless, to map to the classical system of 9 alignments:
\begin{itemize}
\item Lawful characters are between 1.0 and 0.33 on the lawfullness scale, Neutral characters are between 0.3325 and -0.33, Chaotic characters are between -0.33 and -1.0 on the lawfullness scale.
\item Good characters are between 1.0 and 0.33 on the goodness scale, Neutral characters are between 0.33 and -0.33, Evil characters are between -0.33 and -1.0 on the goodness scale.
\item For the sake of convenience and simplicity, at some places, I will still refer to the classical 9-alignment system, and will detail each nine of them below:
\end{itemize}
\subsection{Lawful Good}
\subsection{Neutral Good}
\subsection{Chaotic Good}
\subsection{Lawful Neutral}
\subsection{True Neutral}
\subsection{Chaotic Neutral}
\subsection{Lawful Evil}
\subsection{Neutral Evil}
\subsection{Chaotic Evil}
