\section{Spells}
For characters that rely on magic, \textbf{spells} are powers, abilities and incantations that the spellcaster can rely on for all kinds of purposes, such as healing a wounded ally, damaging enemies, or even completely mundane things, like making menial labour less demanding. In the World of Artograch universe, magic is subdivided by two dimensions. The first being the question whether it is Arcane or Clerical:
\begin{itemize}
\item \textbf{Arcane Magic}, also known as \textit{Profane Magic} is the type of magic utilized by Magicians, Warlocks, Witches and Wizards. This kind of magic utilizes the powers of the spellcaster itself, relying on no aid from deities. Maximum mana depends on the character's Intelligence attribute. When low on mana, the character can use a forbidden technique to consume their own life force or calories to fuel their mana. Arcane Magic tends to influence its spellcaster in various ways, with Dark Magic being known for being an addictive and corrupting influence on its user. Arcane Magic is also known for taking its toll on the spellcaster's physical body.
\item \textbf{Clerical Magic} is the type of magic where the character is granted their powers from the deities they worship, making their magic partially faith-fueled. This comes with both advantages and disadvantages. On the positive side, Clerical Magic does not take its toll on the user, and allows mana to regenerate even during battle \textemdash additionally, while spells can still be learned via books and scrolls, users of Clerical Magic have a tendency to spontaneously learn new spells out of nowhere. On the negative side, their selection of spells is limited by their religion's values \textit{(Clerics affiliated with a Light-oriented religion are limited to healing, anti-undead and anti-demonic spells, for example, with very limited destructive capabilities against foes of other types)}, and they cannot consume their own life forces or calories to regenerate mana. For users of Clerical Magic, their maximal mana depends on their Willpower attribute. To compensate for the limited \textemdash if not nonexistent \textemdash repertoire of destructive spells, users of Clerical Magic tend to compensate by also being capable warriors, especially skilled at melee combat. 
\item \textbf{Nature Magic} is a subvariant of Clerical Magic used by Druids and Rangers, who draw their powers from the \textit{"forces of nature"} as much as they do from their deities: they absorb magical energies while doing deeds the favour nature, such as planting trees, taking care of flowers and feeding wild animals. Another trait of Nature Magic is that it also has the diversity that Arcane Magic has, rather than being limited to a certain category of spells. Druidic Magic also comes with an impressive roster of unique spells that either allow the spellcaster to take the shape of a wild animal or to harness the power of nature by turning the wilderness against the enemy.
\end{itemize}
The other dimension would be the nature of each individual spell itself:
\begin{itemize}
\item \textbf{Basic Spells:} Spells that literally every single spellcaster possesses, without exceptions. This school of magic contains only three spells: Telekinessis \textit{(moves objects and living creatures alike, can be used to move the caster itself, with other potential uses being pushing, pulling, choking, levitation, etc.)}, Lighting \textit{(illuminating dark areas)} and Energy Bolt \textit{(a bolt or arrow composed of nothing but magical energy, the damage inflicted ranging anywhere from harmless to completely lethal, depending on the power of the spellcaster)}.
\item \textbf{Light Magic:} focuses on healing wounds, curing diseases, uncursing people and items alike, blessing people, this school of magic only has a limited roster of spells of destructive nature: the destruction \textit{(or scaring away of)} of undead and demons.
\item \textbf{Dark Magic:} focuses on bringing pain and misery upon the enemy, curses, diseases, poisons, resurrection of the dead by unholy means \textit{(necromancy)}, just like Light Magic, Dark magic only has limited amount of spells that really do cause direct damage, and they often do via slow and cruel means, such as blight and strangulation.
\item \textbf{Elemental Magic:} Making use of the elements, Elemental Magic is usually divided in two or four categories, depending on the school of thought. Some prefer subdividing it into Destruction Magic and Summoning Magic \textit{(the names speak for themselves)}, while others prefer subdividing it based on the four elements \textit{(Earth Magic, Air Magic, Fire Magic, Water Magic)}.
\item \textbf{Utility Magic:} Containing various other spells that cannot be categorized into any of the aforementioned schools of magic, such as spells that aid Lockpicking, spells that lock doors magically, Teleportation, etc. Out of the three Basic Spells, Telekinessis and Lighting can be considered examples of Utility Magic.
\end{itemize}
